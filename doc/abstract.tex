% -*-LaTeX-*-

\begin{abstract}

A number of applications would seem to benefit from storing data ``in
the cloud''---that is at application storage providers such as Amazon
S3.  The cloud model is appealing because it relieves end users from
the need to administer storage systems and worry about backups.
Furthermore it avoids inconsistencies; for example users who store
mail ``in the cloud'' using services such as Gmail can get an
up-to-date consistent view of their mailbox from any PC with Internet
connectivity.

However, the cloud model has a dark size.  Large storage providers
present a massively central point of failure:  if everyone depended on
a handful of services such as Amazon S3, then an outage could cause
untold damage.  Furthermore, storage providers must monetize their
services, which potentially drives them to act against users'
interests by charging fees or, in the case of unencrypted services
such as email, inserting advertising or even violating privacy.

We advocate a different model called ``personal clouds,'' in which
users store data in a reliable cloud, but clouds are built out of each
user's own storage devices.  We present a network file system, pFS,
that realizes this personal cloud model.  pFS exploits the fact that
with the proliferation of devices, users have more and more storage
capacity and redundancy.  pFS is designed to keep a loosely consistent
file system replicated across a sufficient number of devices that data
loss due to hardware failure is highly improbable.  Because of its
week consistency model, pFS achieves good performance by updating
files locally and propagating the changes to other replicas
asynchronously.

\end{abstract}

% Local Variables:
% tex-main-file: "main.ltx"
% tex-command: "make;:"
% tex-dvi-view-command: "make preview;:"
% End:
