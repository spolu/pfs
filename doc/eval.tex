% -*-LaTeX-*-

\section{Evaluation}
\label{sec:eval}

%
% Focus on the evaluation of libpfs : is it really local speed
% Evaluation of PFSD depends on the technology that is intended to be used
% libpfs evaluation :
%    eval tech details / env
%    benchmark : LFS + small Write (relinking + meta_data updt) overhead
% Faster than NFS_local : Definitely usable. 
% Propagation happens asynchronously
% without the user knowing + sequentially

This section presents an evaluation of libpfs performance. Our goal is
to show that pFS provides good local performances resulting in a
seamless user experience, updates begin propagated
ayncrhonously. We compare the performances of pfs, ext3, Fuse only (We
implemented a trivial Fuse based file system replicating all calls
directly to the local file system), and NFS (Protocol v.3 over LAN)
over a set of microbenchmarks. The machine used is ..., running Ubuntu
Server 8.04 distribution.  The microbenchmarks we use are the ones used
for the evaluation of LFS~\cite{rosenblum:lfs}. The small file
benchmark (LFS\_S) consists of the creation of 100000 small (4096 bytes)
files, the access for reading of those files, and finally the deletion
of those files. We augmented the small file benchmark with a small file
write benchmark, which opens for writing and append a few bytes to the
set of files in order to expose the case where libpfs incurs the
most overhead (mainly, an extra {\tt link} call), as described in
section \ref{sec:impl}. The large file benchmark (LFS\_L) consists of
writing sequentially 30000 blocks of 4096 bytes, reading them
sequentially, re-writing them randomly and finally re-reading them
randomly.





\endinput

The actual implementation of pFS does not focus on performance. The
way pFS has been designed would be really adapted to a log-structured
file system. We believe that it is possible drastically improve the
performance by implementing pFS as a log-structured file-system.

% Local Variables:
% tex-main-file: "main.ltx"
% tex-command: "make;:"
% tex-dvi-view-command: "make preview;:"
% End:
