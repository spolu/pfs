% -*-LaTeX-*-


% Restate Intro Model
%     Context / Evolution
%       - Increase of Mobile devices capabilities
%       - At Home data servers
%            (Game Console, Networked Hard Drive)
% Usage case
%     Show increased flexibility over Centralized
%     Architectures
% Extension to / Interplay with Cloud Computing
%     Cloud Computing infrastructure as pFS devices
%     Allow to provide services not 

\section{Personal Clouds}
\label{sec:model}


It appears that mobile devices (PDAs, smartphones, digital audio
players) with connectivity (Wifi, BlueTooth, USB), carried by users
wherever they go, can be used as a substitute to an always available
central infrastructure for update propagation. The connection patterns
between users devices including their mobile devices appear to be more
frequent than the ones offered by a central infrastructure. They are
also more efficient since they take advantage of the higher bandwidth
of physically transported devices.

Based on this observation, our main contribution, is the design of a
flexible versioning system that does not rely on the existence of a
central server, where conflicts are resolved on every devices
participating in the system. Such versioning allows update propagation
between any couple of devices as network conditions permit.

Our intend with pFS is to provide a distributed storage system where
data is cached locally and seamlessly propagated asynchronously. Such
``personnal cloud'' can provide the location independent layer needed
to overcome the management hassle incurred by the growing number of
devices a user has to deal with on a daily basis.

\subsection{Usage Case}

Let's walk through a typical usage case of the ``personal cloud''
infrastructure we propose. User \emph{X} finishes editing his holiday
video at home. The video is stored on his desktop computer, but also
pushed to his digital player via USB. The next day, in the train to
work, without connectivity, he makes a few change on his laptop to his
quarterly report. The update is propagated via bluetooth to his cell
phone. Arriving at work, he plugs in his media player, and enjoy the
video on a widescreen with his colleagues, while the updates
he made on the train are seamlessly uploaded via bluetooth from his
cell phone to his computer at work. He can then work locally on the
latest copy of his report without even booting his laptop. While
his son is using the desktop computer at home to play a game after
school, the updates he made during the day are automatically
fetched. When \emph{X} get back home, he is able to keep working on an
up-to-date copy of its data even in the case he forgot his laptop.

As described above, the ``personal cloud'' infrastructure allows to
take advantage of every devices to propagate data. The cell phone can
be used when connectivity is lacking, and there is no other way to
communicate directly with another device. Any device can be used as a
mean of propagation when the benefit of physically transported devices
overcome the time needed to transmit large sets of data on WAN.

\subsection{Extended Usage}

We believe that such personnal clouds should complement the use of
remote computing and storage clouds. The principles we expose are
compatible with the integration of remote servers as nodes in one's
personnal cloud for providing services that are not available on
conventional devices, such as web-based access and modification of
data.

Our model of ``personnal cloud'' can also be used to spawn devices
owned by different users to ease the collaboration process on files that
otherwise need to be sent back and forth between the participants.

Moreover, pFS represents a solution to back-up since chances are low
for someone to loose all of its devices at the same time. Even if the
mobile devices are used as a way to propagate updates, their loss does not
jeopardize the system since all the data is present on the last device
that connected with it.

Finally, the goal of pFS is to provide a distributed storage system
where the user data is available on all his devices. It should provide
ubiquitous access to up-to-date versions of any ressource without
relying exclusively on a centralized, potentially untrusted (doubtly
free) infrastructure.


\endinput

\subsection {Existing Models}

As we are writing, new services are being introduced to the
market : they are distributed storage systems, with support for
disconnected operations. A local copy of the files is kept on every
device participating in the system. These products results in a user
experience close to what we are aiming at with pFS. Nevertheless, these
systems rely on the existence of a central server, always available,
where conflicts are detected. Figure \ref{OthModel} shows a diagram of
such existing solutions.

\begin{figure}[ht]
\begin{center}
  \includegraphics [scale=0.4] {img/other_model}
  \caption{\label{OthModel}
    {\small Existing models : cloud infrastructure with local
      caches. The circles represent high capacity devices, the diamond
      represents a mobile devices as smartphones or digital players,
      and the square represents the cloud infrastructure always
      available, in charge of resolving conflicting updates. Black
      arrows represent the connection and update propagation patterns
      while the grey arrows represent updates made to the user data.}}
\end{center}
\end{figure}

Unfortunately, we believe that the need for a central server has a few
drawbacks. The end-user has to trust the company providing the cloud
infrastructure for storing \emph{all} his data. Such models might
result into closed infratstructes tighted to the storage cloud
they rely on, where compatibility or migration between different
service providers might become difficult.

\subsection {pFS Personnal Cloud Model}

In such
framework, users' mobile devices, even if considered as any other
devices by the system, become a very common relay for updates. Figure
\ref{PfsModel} shows a diagram of pFS model.

\begin{figure}[ht]
\begin{center}
  \includegraphics [scale=0.4] {img/pfs_model}
  \caption{\label{PfsModel} {\small pFS model : personnal
      cloud. Updates are propagated between any couple of
      devices. The mobile devices appear as efficient relays 
      for updates.}}
\end{center}
\end{figure}




\subsection {Mobile Devices}

In the following sections, we assume that mobile devices as smartphones and
digital players have enough storage capacity for holding all of a
user's data. This might become true in a near future but is certainly
not today. We acknowledge this fact and leave as future work the
design of a caching policy for the mobile devices as described in section 
\ref{sec:futwk}.


% Local Variables:
% tex-main-file: "main.ltx"
% tex-command: "make;:"
% tex-dvi-view-command: "make preview;:"
% End:
