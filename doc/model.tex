% -*-LaTeX-*-

\section{Personal Clouds}
\label{sec:model}

As we are writing, new services are being introduced to the
market : they are distributed storage systems, with support for
disconnected operations. A local copy of the files is kept on every
device participating in the system. These products results in a user
experience close to what we are aiming at with pFS. Nevertheless, these
systems rely on the existence of a central server, always available,
where conflicts are detected. Figure \ref{OthModel} shows a diagram of
such existing solutions.

\begin{figure}[ht]
\begin{center}
  \includegraphics [scale=0.5] {img/other_model}
  \caption{\label{OthModel}
    {\small Existing models : cloud infrastructure with local
      caches. The circles represent high capacity devices, the
      diamond represents a mobile smartphone , and the square
      represents the cloud infrastructure always available, in charge
      of resolving conflicting updates. Black arrows represent the
      connection and update propagation patterns while the grey arrows
      represent updates made to the user data.}}
\end{center}
\end{figure}

Unfortunately, we believe that the need for a central server has a few
drawbacks. The end-user has to trust the company providing the cloud
infrastructure for storing \emph{all} his data. Such models might
result into closed infratstructes tighted to the storage cloud
they rely on, where compatibility or migration between different
service providers might become difficult.

Our main contribution, is the design of a flexible versioning system
that does not rely on the existence of a central server, where
conflicts are resolved on every devices participating in the
system. Such versioning allows update propagation between any couple
of devices participating in the system. Especially, we realized that
using mobile smartphones for relaying updates among the different
devices resulted in efficient update propagation patterns. Figure
\ref{PfsModel} shows a diagram of pFS model.

\begin{figure}[ht]
\begin{center}
  \includegraphics [scale=0.5] {img/pfs_model}
  \caption{\label{PfsModel} {\small pFS Model : personnal
      cloud. Updates are propagated between any couple of
      devices. Especially the mobile smartphone appears as an
      efficient relay for updates. }}
\end{center}
\end{figure}

We think that such personnal clouds should complement the use of
remote computing and storage clouds. Our framework is perfectly
compatible with the use of remote servers as nodes in one's personnal
cloud for providing services that are not available on conventional
devices, such as online availability of data. pFS also provides a
solution to back-up since chances are low for someone to loose all of
its devices at the same time. Even if the smartphone is used as a way to
propagate updates, its loss does not jeopardize the system since all
the data is present on the last device that connected with
it. Finally, the goal of pFS is to provide a distributed storage
system where updates happen locally, and are propagated asyncrhonously
between every devices participating in the system without relying
exclusively on a centralized, potentially untrusted (doubtly free)
infrastructure.


% Local Variables:
% tex-main-file: "main.ltx"
% tex-command: "make;:"
% tex-dvi-view-command: "make preview;:"
% End:
