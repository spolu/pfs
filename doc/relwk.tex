% -*-LaTeX-*-

\section{Related Work}
\label{s:related}

%% Ficus, Coda, P2P filesystems, Bayou, CVS, Git,

pFS was inspired by a number of recent commercial products that follow
the cloud model.  DropBox~\cite{houston:dropbox} creates a remote file
system stored on Amazon S3 and keeps a copy of it locally on every
device involved.  Disconnected operation is permitted, but conflict
detection must take place on DropBox's servers.  Moreover, update
propagation must also go through a remote, centralized infrastructure,
preventing devices from leveraging fast local-area network bandwidth
or other communication links such as Bluetooth or USB\@.  The same
approach is used by .Mac~\cite{apple:mac}.  Live
Mesh~\cite{microsoft:livemesh} also appears to have a similar
architecture to these other products, though unfortunately we don't
have enough details on its implementation.  However, the fact that it
limits the space available on a ``mesh'' suggests that Microsoft's
central storage infrastructure may be a limiting factor.

The goal of pFS is to provide a similar user experience to these
services, yet without relying on a centralized infrastructure.  Thus,
users can make full use of the storage capacity and network bandwidth
supported by their devices, without paying any fees, viewing any
advertising, losing any privacy, or unnecessarily subjecting
themselves to storage quotas.  Our thesis is that not can many of the
well-known concepts in distributed file systems be readily adapted to
disintermediate these central services, but that hardware trends and
the specific usage model allow for significant design simplifications
compared to previous work.

The most relevant previous file systems are Ficus~\cite{page:ficus}
and CODA~\cite{kistler:coda}, both of which support replication of
servers and disconnected operation.  Compared to these systems, pFS
achieves simplicity by eliminating the distinction between clients and
servers, eliminating the need for caching and hoarding (using only
replication), assuming only a modest number of clients (on the order
of a dozen), eliminating a synchronous/consistent mode of operation,
exposing conflicts in such a way that they can be resolved by software
entirely outside of the file system, and assuming that reconciliation
can happen over high-bandwidth links by transporting devices (thereby
eliminating the need for techniques such as trickle reintegration).

\cite{reiher:resolving}

BlueFS~\cite{nightingale:bluefs}
EnsemBlue~\cite{nightingale:bluefs}



Unison~\cite{balasubramanian:unison}
Rumor~\cite{guy:rumor}
Tra~\cite{cox:tra}

Bayou~\cite{petersen:flexible-update}
PRACTI~\cite{belaramani:practi}

Pastiche~\cite{cox:pastiche,nguyen:friendstore}


% Local Variables:
% tex-main-file: "main.ltx"
% tex-command: "make;:"
% tex-dvi-view-command: "make preview;:"
% End:
