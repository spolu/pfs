% -*-LaTeX-*-

\section{Related Work}
\label{s:related}

%% Ficus, Coda, P2P filesystems, Bayou, CVS, Git,

pFS was inspired by a number of recent commercial products that follow
the cloud model.  DropBox~\cite{houston:dropbox} creates a remote file
system stored on Amazon S3 and keeps a copy of it locally on every
device involved.  Disconnected operation is permitted, but conflict
detection must take place on DropBox's servers.  Moreover, update
propagation must also go through a remote, centralized infrastructure,
preventing devices from leveraging fast local-area network bandwidth
or other communication links such as Bluetooth or USB\@.  The same
approach is used by .Mac~\cite{apple:mac}.  Live
Mesh~\cite{microsoft:livemesh} also appears to have a similar
architecture to these other products, though unfortunately we don't
have enough details on its implementation.  However, the fact that it
limits the space available on a ``mesh'' suggests that Microsoft's
central storage infrastructure may be a limiting factor.

The goal of pFS is to provide a similar user experience to these
services, yet without relying on a centralized infrastructure.  Thus,
users can make full use of the storage capacity and network bandwidth
supported by their devices, without paying any fees, viewing any
advertising, or losing any privacy.  Our thesis is that not can many
of the well-known concepts in distributed file systems be readily
adapted to disintermediate these central services, but that hardware
trends


. Our
goal was to take advantage of evolution in hardware to design a simple
system yet providing benefits comparable to more complex system
relying on highly available central infrastructures. pFS weaker
consistency and its simplicity makes it also portable to a very wide
type of devices relying on any type of communication channel.


Unison~\cite{balasubramanian:unison}
Rumor~\cite{guy:rumor}
Tra~\cite{cox:tra}

Bayou~\cite{petersen:flexible-update}
PRACTI~\cite{belaramani:pract}

Pastiche~\cite{cox:pastiche,nguyen:friendstore}


% Local Variables:
% tex-main-file: "main.ltx"
% tex-command: "make;:"
% tex-dvi-view-command: "make preview;:"
% End:
