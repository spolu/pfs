% -*-LaTeX-*-

\section{Related Work}

%% Ficus, Coda, P2P filesystems, Bayou, CVS, Git,

pFS relies on well-known concepts in distributed file systems. Our
goal was to take advantage of evolution in hardware to design a simple
system yet providing benefits comparable to more complex system
relying on highly available central infrastructures. pFS weaker
consistency and its simplicity makes it also portable to a very wide
type of devices relying on any type of communication channel.

Lately, the idea of storing every file locally has appeared in
commercial products. DropBox~\cite{houston:dropbox} creates a remote
file system stored on Amazon S3 and keeps a copy of it locally on
every devices involved. Disconnected operations are therefore
permitted, but conflicts detection happens on their servers, and
update propagation have to go through their central infrastructure and
thus cannot leverage on devices locality or other communication links
such as Bluetooth or USB. The same infrastructure is used by
.Mac~\cite{apple:mac}. Unfortunately, we don't have enough details on
Live Mesh~\cite{microsoft:livemesh}, but the limit imposed to the space
available on a ``mesh'' tends to show that its central storage
infrastructure has been given a critical role in the devices
synchronization process. pFS design has been inspired by these
products, and aims at providing the same user experience, relying on
the user's devices storage capacity instead of a central
infrastructure, and therefore allowing more flexible connection
patterns.


% Local Variables:
% tex-main-file: "main.ltx"
% tex-command: "make;:"
% tex-dvi-view-command: "make preview;:"
% End:
