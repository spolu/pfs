% -*-LaTeX-*-

\section{Related Work}
\label{s:related}

%% Ficus, Coda, P2P filesystems, Bayou, CVS, Git,

pFS was inspired by a number of recent commercial products that follow
the cloud model.  DropBox~\cite{houston:dropbox} creates a remote file
system stored on Amazon S3 and keeps a copy of it locally on every
device involved.  Disconnected operation is permitted, but conflict
detection must take place on DropBox's servers.  Moreover, update
propagation must also go through a remote, centralized infrastructure,
preventing devices from leveraging fast local-area network bandwidth
or other communication links such as Bluetooth or USB\@.  The same
approach is used by .Mac~\cite{apple:mac}.  Live
Mesh~\cite{microsoft:livemesh} also appears to have a similar
architecture to these other products, though unfortunately we don't
have enough details on its implementation.  However, the fact that it
limits the space available on a ``mesh'' suggests that Microsoft's
central storage infrastructure may be a limiting factor.

The goal of pFS is to provide a similar user experience to these
services, yet without relying on a centralized infrastructure.  Thus,
users can make full use of the storage capacity and network bandwidth
supported by their devices, without paying any fees, viewing any
advertising, losing any privacy, or unnecessarily subjecting
themselves to storage quotas.  Our thesis is that not can many of the
well-known concepts in distributed file systems be readily adapted to
disintermediate these central services, but that hardware trends and
the specific usage model allow for significant design simplifications
compared to previous work.

The most relevant previous file systems are Ficus~\cite{page:ficus}
and CODA~\cite{kistler:coda}, both of which support replication of
servers and disconnected operation.  Compared to these systems, pFS
achieves simplicity by eliminating the distinction between clients and
servers, eliminating the need for caching and hoarding (using only
replication), assuming only a modest number of clients (on the order
of a dozen), eliminating a synchronous/consistent mode of operation,
exposing conflicts in such a way that they can be resolved by software
entirely outside of the file system, and assuming that reconciliation
can happen over high-bandwidth links by transporting devices (thereby
eliminating the need for techniques such as trickle reintegration).
pFS also uses organizes its representation of the file system around
directory entries rather than files, which avoids the need to handle
update/update conflicts and name conflicts separately, as in previous
systems~\cite{reiher:resolving}.

BlueFS~\cite{nightingale:bluefs} and a follow-on project
EnsemBlue~\cite{nightingale:bluefs} are both file systems targeted at
mobile devices.  These projects also address the problems of cache and
power management, performance optimization in the face of different
storage media, and namespace management, none of which pFS
specifically addresses.  Like pFS, BlueFS takes advantage of users
transporting storage devices to move data directly between pairs of
clients.  Unlike pFS, BlueFS still requires a centralized server.  The
authors seem to view servers as providing backup for mobile devices
more than vice versa.

Other peer-to-peer file systems have been proposed with no central
server, notably Ivy~\cite{muthitacharoen:ivy}.  However, a big
difference between pFS and Ivy is that pFS replicates a file system
only on devices belonging to a particular user, whereas Ivy spreads
the data over potentially huge numbers of machines owned by different
people.

A closer analog to pFS's peer-to-peer architecture is distributed
revision control systems such as BitKeeper and git~\cite{git}.  These
systems have an egalitarian structure that eliminates the concept of a
``master'' repository---all repositories are equal, and can
synchronize with each other pairwise.  However, while repositories
store complete modification histories, pFS only retains enough recent
versions of each file to ensure the ``no lost updates'' property.  In
this sense, pFS is closer to directory mirroring tools such as
Unison~\cite{balasubramanian:unison}, Rumor~\cite{guy:rumor}, and
Tra~\cite{cox:tra}.

pFS is related to several other distributed systems that are not file
systems.  Bayou~\cite{petersen:flexible-update} was designed to handle
concurrent updates to objects in a disconnected fashion, but was
explicitly intended not to be a file system so as to force
applications to provide conflict resolution code.
PRACTI~\cite{belaramani:practi} provides topology independence like
pFS, but also has such features as partial replication which pFS
doesn't need.  Finally peer-to-peer backup systems such as
Pastiche~\cite{cox:pastiche} \cite{nguyen:friendstore} address the
backup problem in a different way from pFS, by allowing people to use
each others' spare disk space for backup.



% Local Variables:
% tex-main-file: "main.ltx"
% tex-command: "make;:"
% tex-dvi-view-command: "make preview;:"
% End:
