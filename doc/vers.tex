% -*-LaTeX-*-

\section{Versioning}
\label{sec:vers}

This section describes pFS's model for reconciling different versions
of files when receiving updates asynchronously.

\begin{figure*}
\centerline{\input{pfs_struct}}
\caption{PFS data structures}
\label{fig:struct}
\end{figure*}

\subsection{Terminology}

We use the term \emph{device} to mean any hardware running the pFS
software.  Though pFS currently only runs on Unix machines, we
envision porting it to other types of hardware such as PDAs, cell
phones, and digital audio players.

A pFS \emph{file system} is a directory tree replicated on one or more
devices.  Each file system has a unique ID generated at creation time.
We use the term \emph{replica} to denote a copy of a particular file
system stored by a device.  Note that one device may store replicas of
multiple file systems.

Each replica is also owned by a particular \emph{user}.  (Usually all
replicas on the same device have the same user, but this need not be
the case.)  Each replica has a unique name, usually given by the
tuple $\langle \mathrm{file\ system}, \mathrm{device}, \mathrm{user}\rangle$.

Each file system has associated metadata consisting of an access
control list (list of users allowed read or read-write access), and a
set of replicas currently storing the file system.  File systems can
thus be shared between users, but each user has a special private file
system

We denote by $sd_{d}$ a storage space of capacity $S(sd_{d})$
located on the device $d$. A group is a set of $sd$s and an
access control list:
\begin{center}
$g = (\{sd_{d}\}, \text{ACL})$
\end{center}
Note that a given device can have multiple $sd$s and that $sd$s are
not shared among different groups. But a $sd$ is provided and intended
to be used by a given user $u$. Therefore each $sd$ is given a
readable name :
\begin{center}
$\text{name}(sd) = \text{user}.\text{device}$. 
\end{center}
Moreover, Each user has a special group $me(u)$ on which he has
exclusive access:
\begin{center}
$me(u) = (\{sd_{d}\}, \text{u:rw})$
\end{center}

\subsection{Disconnected Operations}

pFS makes no assumption concerning network partitioning and
connectivity. We model connectivity as connections happening between
two $sd$s belonging to the same group and allowing all data and
metadata concerning a given ressources to be exchanged. We denote such
connection between $sd_{1}$ and $sd_{2}$ concerning ressource $f$ as
$c(sd_{1}, sd_{2}, f)$.

\begin{figure}[t]
\begin{center}
{\tt \small
\begin{verbatim}
struct pfs_dir
{
  char id [PFS_ID_LEN];
  uint32_t entry_cnt;
  struct pfs_entry ** entry;
};
struct pfs_entry
{
  char name [PFS_NAME_LEN];
  uint32_t main_idx;
  uint32_t ver_cnt; 
  struct pfs_ver ** ver;
};
enum pfs_entry_type {
  PFS_DEL = 0,
  PFS_FIL = 1,
  PFS_DIR = 2,
  PFS_SML = 3 
};
struct pfs_ver
{
  uint8_t type;
  char dst_id [PFS_ID_LEN];
  struct pfs_vv * vv;
};
struct pfs_vv
{ 
  char last_updt [PFS_ID_LEN];
  uint32_t len;
  char ** sd_id;
  uint32_t * value;
};
\end{verbatim}
}
\end{center}
\caption{\label{MemStruct}
{\small In-memory data structures for directory, entries and version
  vectors.}}
\end{figure}

\begin{table*}[t]
\begin{center}
  \begin{tabular}[t]{|l||cc|cc|cc|}
    \hline
    \emph{ops} & $sd_{1}$ & & $sd_{2}$ & & $sd_{3}$ & \\
    \hline

    $sd_{1} : f_{\text{create}}$ & 
    $\langle 1,0,0\rangle$ & &
     & &
     & \\

    $c(sd_{1},sd_{2},f)$ &
    $\langle 1,0,0 \rangle$ & &
    $\langle 1,0,0 \rangle$ & &
     & \\

    $sd_{2} : f_{\text{modif}}$ & 
    & &
    $\langle 1,1,0 \rangle$ & &
     & \\

    $sd_{1} : f_{\text{modif}}$ &
    $\langle 2,0,0 \rangle$ & &
    & &
     & \\

    $c(sd_{2},sd_{3},f)$ &
    & & 
    $\langle 1,1,0 \rangle$ & &
    $\langle 1,1,0 \rangle$ & \\

    $c(sd_{1},sd_{3},f)$ &
    $\langle 2,0,0 \rangle$ & $\langle 1,1,0 \rangle$ &
    & &
    $\langle 1,1,0 \rangle$ & $\langle 2,0,0 \rangle$ \\

    $sd_{3} : f_{\text{modif}}$ &
    & &
    & & 
    $\langle 1,1,1 \rangle$ & $\langle 2,0,0 \rangle$ \\

    $c(sd_{2},sd_{3},f)$ &
    & &
    $\langle 1,1,1 \rangle$ & $\langle 2,0,0 \rangle$ & 
    $\langle 1,1,1 \rangle$ & $\langle 2,0,0 \rangle$ \\

    $sd_{2} : f_{\langle 1,1,1 \rangle \leftarrow \langle 2,0,0 \rangle}$ &
    & &
    $\langle 2,2,1 \rangle$ & &
    & \\

    $c(sd_{1},sd_{2},f)$ &
    $\langle 2,2,1 \rangle$ & &
    $\langle 2,2,1 \rangle$ & & 
    & \\

    $c(sd_{1},sd_{3})$ &
    $\langle 2,2,1 \rangle$ & &
    & &
    $\langle 2,2,1 \rangle$ & \\

    \hline
  \end{tabular}
\end{center}
\caption{\label{VVEx}
  {\small Evolution of version vectors for a file $f$ modified from
  three different $sd$s. For each $sd$ we show the main version
  vector relative to that $sd$ fisrt. We show the version vector list
  for each $sd$ only when it is modified.
  Here the maximum number of
  coexisting versions for $f$ is 2. We could have reached the worst
  case, that is 3, by modifying $f$ from the three different $sd$s
  without any connection occuring between modifications.}}
\end{table*}

\subsection{Version Vector}

pFS does not maintain a versioning of the files and directories but a
versionning of the directory entries. It allows much more flexibility
to handle cases where two different ressources (e.g: a file and a
directory) have been given the same name on two different $sd$s.  pFS
associates with each file and directory a global unique identifiant
generated on creation by applying a 64-bits md5 cryptographic hash
function on the concatenation of the identifiant of the $sd$ where the
ressource is created and an incremented counter maintained on each
$sd$. 

As shown in Figure \ref{MemStruct}, a directory is represented in pFS
by a {\tt pfs\_dir} structure. It is a list of versionned {\tt
  pfs\_entry} structures mapping names to ids. To encompass the notion
of file versioning, each time a file is modified, its id is
regenerated and associated {\tt pfs\_ver} structure updated.

Each version of an entry is associated with a version vector as
defined by Parker \emph{et al.}~\cite{parker:inconsistency} from which we
borrow the following definition replacing ``file'' by ``entry'' and
``sites'' by ``$sd$s'' :

\begin{definition}
A version vector for an entry $e$ is a sequence of $n$ pairs, where
$n$ is the number of $sd$s at which $e$ is stored. The $i$th pair
$(sd_{i}, v_{i})$ gives the index of the latest version of $e$ made at
$sd_{i}$. In other words, the $i$th vector entry counts the number
$v_{i}$ of updates to $e$ made at $sd_{i}$.
\end{definition}

When the $sd$s and their order are known or implied, we will represent a
version vector as $\langle v_{1}, ..., v_{n} \rangle$. From now on, we
will designate a versionned entry $e$ by the name it associates with
ids. When talking about entry $f$, we refer to the versionned id that
associates name $f$ with a set of ids pointing to different
ressources.

We say that a version vector $vv^{(1)}$ dominates $vv^{(2)}$ if
$\forall sd_{i}, v_{i}^{(1)} \geq v_{i}^{(2)}$. Conflicts are reliably
detected when it exists two version vectors $vv^{(1)}, vv^{(2)}$ such
that $vv^{(1)}$ does not dominate $vv^{(2)}$ and $vv^{(2)}$ does not
dominate $vv^{(1)}$. 

\subsubsection{Conflicts}

When a conflict is detected, pFS does not attempt to resolve it
automatically. It relies on concepts introduced by Gifford 
\emph{et al.}~\cite{gifford:sfs} 
to provide feedback to users or applications by generating two
different file names when reading the directory for the two different
versions of the entry. When a conflict is detected for an entry $f$,
two names pointing to two different ressources are listed : 
\begin{center}
{\tt name($sd_{i}$):f} \\
{\tt name($sd_{j}$):f}
\end{center}
Where $sd_{i}$ and $sd_{j}$ are the $sd$s where
the last update happened for each version of the entry.

\subsubsection{Merge}

Merging is achieved by using the {\tt rename} system call. Once all
information from the two different versions has been inserted by the
user or an application into one of the versions of the entry,
invoking:
\begin{center} 
{\tt rename (name($sd_{j}$):f,name($sd_{i}$):f)}
\end{center}
 deletes entry version {\tt name($sd_{j}$):f} along with the
 underlying ressource and updates 
{\tt name($sd_{i}$):f}'s version vector. The underlying content of 
{\tt name($sd_{i}$):f} stay unchanged, since the actual merging
operation has to be done before calling {\tt rename}.
If the two initial version vectors where
$vv^{(i)}=(sd_{k},v_{k}^{(i)})_{k}$ and
  $vv^{(j)}=(sd_{k},v_{k}^{(j)})_{k}$, The resulting version
vector is 
\begin{center}
$vv^{(i \leftarrow j)}=(sd_{k},
max(v_{k}^{(i)},v_{k}^{(j)}) + \delta _{i}^{k})_{k}$
\end{center}
The resulting version vector dominates both original version vectors
and count one update for this merging operation for $sd_{i}$.

Hence, users are able to detect conflicts, resolve them manually or
automatically using third-party application, and merge the two
versions into a final version that supersedes them. All those
operations can be expressed through the classical file
system interface and therefore require absolutely no change to
existing terminals and utilities.

\subsubsection{Main Version}

To avoid the proliferation of entry versions, we introduce the notion
of \emph{main version}. We define an order $<_{sd_{i}}$ relative to
$sd_{i}$ over the versions of a given entry :

\begin{definition}
We say that $vv^{(1)} >_{sd_{i}} vv^{(2)}$ if the update count
relative to $sd_{i}$ is greater in $vv^{(1)}$ than it is in
$vv^{(2)}$. Or if those quantities are equals, if the total number
of updates is greater in $vv^{(1)}$ than it is in $vv^{(2)}$.
\end{definition}
 
We can always determine for each $sd$ a maximum version verctor (by
breaking the ties arbitrarly if two or more candidates are
``equals''). We call the associated entry version the main version for
that $sd$.

When multiple versions of an entry named {\tt f} coexist, we display
the main version as {\tt f} and the other version as before :
\begin{center}
  {\tt f} \\
  {\tt name($sd_{i}$):f} \\
  {\tt ...}
\end{center}
Moreover pFS enforces that only the main version of an entry is
writable, the other ones being accessible read-only.
Therefore, merging process is achieved by inserting into the main
version of an entry all the information contained in another version,
and then calling :
\begin{center}
{\tt rename (name($sd_{j}$):f, f)}.
\end{center}
Under such restrictions, a $sd$ can only be the last updater of its
main version, and the total number of versions seen by a $sd$ for any
given entry is bounded by the number of $sd$s participating in the group.
{\proof
Suppose we have $k$ version vectors, $vv^{(1)},..., vv^{(k)}$ for an
entry $f$ at a given $sd$, and $n$ $sd$s, $sd_{1}, ..., sd_{n}$
participating in the group. A new version vector can only be generated
by updating an entry or merging two entry versions, and the resutling
version vector is necessarely the main version vector of the $sd$
where the update occured. Therefore, for each version vector
$vv^{(i)}$, it exists $l$ such that $sd_{l}$ has $vv^{(i)}$ as main
version vector relatively to the set $vv^{(1)},..., vv^{(k)}$. If it
exists $vv^{(i)},vv^{(j)}$ such that $sd_{l}$ has $vv^{(i)}$ and
$vv^{(j)}$ as main version vector, then one dominates the other and
$i=j$. Thus, $k \leq n$. \qed
}

Note that the set $vv^{(1)},..., vv^{(k)}$ can differ from one $sd$ to
the other if all updates are not yet propagated to all $sd$s. Table \ref{VVEx}
shows the evolution of the version vectors associated with an
entry updated from three different $sd$s.

\subsection {${\tt pfs\_set\_entry}$}

A nice property of this versionning system is that every classical
file system operations can be mapped as a call to a unique
operation {\tt pfs\_set\_entry}. {\tt pfs\_set\_entry} is the atomic
update operation that will be propagated among the $sd$s of a group.
It has a log structure, that is, replaying sequentially a set of update
allows to reconstruct the associated file system tree structure. When
propagated to a $sd$, the updates are applied and can be simply appended to a
log for future propagation to other $sd$s.  
The declaration of the function is as follows:

\begin{center}
{\tt \small
\begin{verbatim}
int pfs_set_entry 
  (struct pfs_instance * pfs,
   const char * grp_id,
   const char * dir_id,
   const char * name,
   const uint8_t reclaim,
   const struct pfs_ver * ver);
\end{verbatim}
}
\end{center}

{\tt pfs\_set\_entry} adds version {\tt ver} (see Figure
\ref{MemStruct} for a declaration of {\tt pfs\_ver} structure) to the
entry {\tt name} in the directory designated by {\tt dir\_id} in the
group designated by {\tt grp\_id}. Note that the directory structure
does not have a name, but is simply designated by its {\tt
  dir\_id}. The mapping from the name to {\tt dir\_id} being itself
represented as an entry in the parent directory.  {\tt reclaim} is
used to determine wether or not ressources pointed by versions that
are dominated by {\tt ver} in entry {\tt name} should be reclaimed or
not. It is particularly usefull for the {\tt rename} operation. Let us
show a few examples of how file system operations map to {\tt
  pfs\_set\_entry} operations. For clarity, we ignore the {\tt
  grp\_id} and {\tt dir\_id} and only show the type of the {\tt ver}
used :

\begin{itemize}
  \item \textbf{open ($f$, O\_CREAT)} : \\
    generate new back storage file and $id$ \\
    {\tt pfs\_set\_entry ($dir\_id$,$f$,(FIL, $id$),1)}

  \item \textbf{close ($f$)} when $f$ is dirty : \\
    regenerate back storage $id'$ \\
    {\tt pfs\_set\_entry ($dir\_id$,$f$,(FIL, $id'$),1)}
    
  \item \textbf{unlink ($f$)} : \\
    {\tt pfs\_set\_entry ($dir\_id$,$f$,(DEL, 0),1)}

  \item \textbf{rmdir ($d$)} : \\
    {\tt pfs\_set\_entry ($dir\_id$,$d$,(DEL, 0),1)}

  \item \textbf{rename ($f_1$, $f_2$)} : \\
    {\tt pfs\_set\_entry ($dir\_id$,$f_1$,(DEL, 0),0)} \\
    {\tt pfs\_set\_entry ($dir\_id'$,$f_2$,(FIL, $id$),1)}
\end{itemize}

We have not shown the version vectors used here. They are generated or
incremented according to the semantics we defined previously. Note
that a deleted entry version persists in the system as being of type
{\tt DEL}. Such conservative technique is needed to handle cases where
an entry might have been deleted on one $sd$ and updated on
another. The {\tt DEL} type has to be treated as any other type of
entry by the versionning system.

Moreover, anytime a {\tt pfs\_set\_entry} update is received from
another $sd$ with an unknow {\tt dir\_id}, it generates a new {\tt pfs\_dir}
with the unknown {\tt dir\_id} that is not inserted in the local tree, but
will allow to complete the received update. Such situation may arise
if the updates are not propagated according to the same order as the one they
have been originally generated with. If all updates are subsequently
propagated, an update relative to the new directory will follow and will
allow its insertion in the local tree. This behavior gives the 
{\tt pfs\_set\_entry} updates a very useful property : \emph{a given set of
updates will result in the same local tree structure regardless of the
order in which the updates are performed.}

Updates are naturally ordered by the Happen-Before relationship, and
one can advocate to keep this order whenever a set of updates is
propagated. Nevertheless, we rely explicitly on the order independance
described above in one precise case : directory removal. Assume
$sd_{1}$ and $sd_{2}$ are disconnected, $sd_{1}$ removes a
directory $d$ and the one file contained in it $f$, while $sd_{2}$
makes a modification to $f$. When $sd_{2}$ will receive $sd_{1}$
updates, $f$ will first exist as two non-dominated versions, a file
and a a deleted entry. Then an attempt will be made to remove
directory $d$ which will fail since $d$ is not empty on $sd_{2}$. Here
$sd_{2}$ has to replay $d$ creation with an increased version
vector. When $sd_{2}$ updates will be propagated to $sd_{1}$, it will
first see the creation of a file $f$ for an unknown {\tt dir\_id}, and
will accept the update and generate a new directory with the unknown
{\tt dir\_id}. Then, it will receive the replayed creation of $d$,
allowing it to reinsert the mapping from name $d$ to {\tt dir\_id}, into
the parent directory. Finally, $d$ maps on every $sd$ to two non-dominated
versions, $(DEL,0)$ as main version for $sd_{1}$ and, $(DIR, dir\_id)$ as main version
for $sd_{2}$.

%QUID : Do you see the little consistency that exists after this
%operation. It has no impact on the logical structure of the
%tree...
%Should I mention it ?

Other solutions could have been provided to solve this directory
removal problem. All other programation of updates we considered would
have implied loosing the log structure of {\tt pfs\_set\_entry}
updates which we believe has to be preserved for its simplicity and
the flexibility it provides for updates propagation.

Thus, {\tt pfs\_set\_entry} is the atomic update operation used by
pFS. Any modification to the file system tree structure can be mapped
to a {\tt pfs\_set\_entry} update. The updates are organized as a
simple order-independent log structure, which provides a great deal of
flexibility and simplicity for updates propagation.

% Local Variables:
% tex-main-file: "main.ltx"
% tex-command: "make;:"
% tex-dvi-view-command: "make preview;:"
% End:
