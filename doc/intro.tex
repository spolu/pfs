% -*-LaTeX-*-

\section{Introduction}

%
%  More storage devices
%    More failures
%    More places your data could be--harder to manage
%  One solution:  The cloud
%    Put all your data at Google or Amazon
%    Alleviates the failure and management problems
%  Limitations of the cloud model
%    Requires you to be on-line (even phones don't always have net access)
%    Potentially uses lots of network bandwidth to the outside world
%    Creates massive single points of failure
%    Creates need to support storage provider business model
%         (e.g., why pay Amazon S3 charges if you don't have to?)
%  But don't really need storage provider's hardware, just management
%    People usually have enough raw storage on devices for their data
%          Storage growing faster than net connectivity (cite Rodrigues)
%    People lose servers, desktop machines, and cell phones,
%          but usually not all at once
%  Suggests an alternative:  Personal clouds
%    Serverless cloud model based on your own devices
%    Example of taking data to work on cell phone
%  Requires a new data model
%    Don't want to copy files manually all the time--too inconvenient
%    Don't want a network file system
%          Synchronous network file system too dependent on network
%          Coda/Ficus still require server which requires maintenance
%    We advocate new type of FS:  Update locally, propagate asynchronously
%  Roadmap
%

Users have an increasing number of devices with sizable data storage.
PDAs, digital audio players, cell phones, and cameras will all soon
provide storage capacities comparable to today's laptops.  The ability
to carry around more storage opens up new opportunities, but also
creates complications.  First, more devices mean more failures.
People are more likely to lose or damage a cell phone or PDA than a
desktop machine; yet when cell phones have 60~GB of storage, they will
likely contain data the owner cannot afford to use.  Second, more
devices mean there are more places any given file may live. Whatever
file a user wants is then less likely to be on the device he or she is
currently using.  Users will thus somehow need to manage their data by
transferring between devices.

One solution to these problems is to store all data ``in the
cloud''---i.e., to keep the primary copy of data at an external
service provider such as Amazon or Google, rather than in a user's own
devices.  This trend is particularly evident with web mail, which many
find preferable to running their own mail servers.  Web mail solves
the synchronization problem of checking mail from multiple clients.
It also saves end users from dealing with server failures.  Web sites
such as Google Docs promise a similar experience for a broader range
of applications.

Unfortunately, the cloud model has a number of limitations.  In many
cases, users need Internet connectivity to access their data.  Yet
devices---even cell phones---do not always have network connectivity.
Moreover, the cloud model unnecessarily consumes precious wide-area
network bandwidth even when two devices on the same network share
files.  Worse yet, widespread dependence on a small number of storage
providers creates a dangerously large central point of failure.
Pakistan recently caused a world-wide blackout of YouTube; there's no
reason someone couldn't similarly take down Amazon~S3.  Finally, the
cloud model requires service providers to recover costs.  Some
applications, such as email, can achieve this through advertising.
Others (such as those based on Amazon S3) require fees to the storage
provider.  In these cases an alternative that does not require fees
would be preferable for users.

Two trends suggest a better alternative to the cloud model for many
users.  First, the raw storage owned by users is plentiful and growing
faster than Internet connection speeds.  From 1990 to 2005, the time
required to transmit a typical home user's hard disk contents over the
wide-area network increased from 0.6 to 120
days~\cite{rodrigues:multi-hop}.  Usable storage capacity of ``the
cloud'' is therefore limited by bandwidth to something much smaller
than local storage.  Second, as users get more devices, they have
access to more failure independence.  Though we often lose servers,
desktop machines, laptops, cell phones, or PDAs, any single person is
highly unlikely to lose all of their devices simultaneously.

These trends suggest an alternative data management model, which we
call \emph{personal clouds}.  A personal cloud is a file system in
which multiple devices each store a local copy of every file.
Modifications propagate between devices opportunistically as network
conditions permit.  For instance, modifications made on a user's home
machine might automatically propagate to her laptop over a local
wireless network; at work the next day the laptop could automatically
push the changes out to a local server.  Though changes can also
propagate over the wide area network, personal clouds exploit the
higher bandwidth of people physically transporting devices to
synchronize much more data than would otherwise be practical.
Furthermore, should any device fail, its replacement can simply
re-initialize itself from any existing copy.


% Local Variables:
% tex-main-file: "main.ltx"
% tex-command: "make;:"
% tex-dvi-view-command: "make preview;:"
% End:
