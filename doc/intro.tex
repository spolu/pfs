% -*-LaTeX-*-

\section{Introduction}

%
%  More storage devices
%    More failures
%    More places your data could be--harder to manage
%  One solution:  The cloud
%    Put all your data at Google or Amazon
%    Alleviates the failure and management problems
%  Limitations of the cloud model
%    Requires you to be on-line (even phones don't always have net access)
%    Potentially uses lots of network bandwidth to the outside world
%    Creates massive single points of failure
%    Creates need to support storage provider business model
%         (e.g., why pay Amazon S3 charges if you don't have to?)
%  But don't really need storage provider's hardware, just management
%    People usually have enough raw storage on devices for their data
%          Storage growing faster than net connectivity (cite Rodriguez)
%    People lose servers, desktop machines, and cell phones,
%          but usually not all at once
%  Suggests an alternative:  Personal clouds
%    Serverless cloud model based on your own devices
%    Example of taking data to work on cell phone
%  Requires a new data model
%    Don't want to copy files manually all the time--too inconvenient
%    Don't want a network file system
%          Synchronous network file system too dependent on network
%          Coda/Ficus still require server which requires maintenance
%    We advocate new type of FS:  Update locally, propagate asynchronously
%  Roadmap
%

Users have an increasing number of devices with sizable data storage.
PDAs, digital audio players, cell phones, and cameras will all soon
provide storage capacities comparable to today's laptops.  The ability
to carry around more storage opens up new opportunities but also
creates several complications.  More devices mean more failures.
People are more likely to lose or damage a cell phone or PDA than a
desktop machine; yet when the cell phone has 60~GB of storage, it will
be more likely to contain irreplaceable data.  Second, more devices
mean there are more places any given file can live, so that whatever
file a user wants is less likely to be on the device he or she is
currently using.  Synchronizing data back and forth between devices
creates its own management hassle.

One solution people appear to be moving towards is storing all data
``in the cloud''---i.e., keeping the primary copy of data at external
service providers such as Amazon or Google rather than in any of a
user's own devices.  This trend is particularly evident with web mail,
which many people find preferable to using their own mail servers
as it allows them to check mail from multiple devices and not worry
about their own servers failing.  Similarly, some people use web
applications such as Google spreadsheet, particularly when sharing the
data with multiple people using multiple devices.

\endinput


on which they need to
access their data ranging from desktop computer to laptop computers
and mobile smartphones, they face the data management problems of
maintaining multiple versions of a file on different devices. They
achieve synchronization by sending email or eventually relying on a
central remote file server. But manual synchronization is very
error-prone, and a central server cannot be accessed when disconnected
from the network. Moreover, not all users are ready to trust any
organization to store their personnal data.

Collaboration is even more error prone. Users are often force to rely
on one person to cenralize all updates in order to keep track of the
most recent version of a ressource. CVS has answered this problem for
developpers, but requires manual intervention from the user, and
setting up a central repository for files.

The growing use of third-party storage and computing ``clouds'', such
as Google Web Services or Amazon S3 clearly shows the need for new
solutions to this problematic. But we believe that they also have
limitations in terms of trust and need for connectivity and could be
complemented by the use of ``personnal clouds'' spawning all of a
user's or a group of users' devices interacting together at the hedge
of the network. If so, mobile devices as well as dekstop computers
will have to coordinate their interaction in the context of
disconnected operations and disparate storage capacities, creating new
needs for synchronization and replication that are not answered by
conventional systems. Building a unified storage and naming system in
this context is a first step to such ``personnal clouds''.

pFS is a decentralized storage system that makes no assumption on
connectivity or network partitioning and expose data through a
standard file system interface. pFS replicates all or a
subset of the user's data on all devices. Updates to the ressources
take place locally and the propagation of those updates to other
devices happen asynchronously. In such context, concurrent updates to
the same ressource becomes an important issue since we have no
centralized instance to synchronize them. We introduce a new
versioning system based on Parker \emph{et al.}'s version vectors
~\cite{Parker1983} that limits the number of coexisting version of a
same ressource while ensuring the detection of all conflicts.

When it comes to mobile devices with smaller storage capacities than
dekstop computers, with replication comes eviction. pFS apply a local
eviction policy on files when not enough space is left on a device
while ensuring that no data is lost globally. Finally, pFS makes no
assumption concerning the format of the files and rely on the user or
its application to merge different versions manually or automatically.
It also provides an interface to influence the decisions made by the
eviction policy.

% Local Variables:
% tex-main-file: "main.ltx"
% tex-command: "make;:"
% tex-dvi-view-command: "make preview;:"
% End:
