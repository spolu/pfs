% -*-LaTeX-*-

\section{Discussion and Future Work}
\label{sec:futwk}

There is no final design for pfsd since it is highly dependent on
which type of device and which type of communication channel it relies
on. This section is a survey of lessons we have learned from designing
pfsd and directives we have identified to iterate on our implementation.

\paragraph {Naming service}
In the context of IP a naming service is necessary to enable the
devices to discover each others' address from their unique names. We
used a central server, but DNS could also be used as a naming service
if public IPs are available. A naming service is not always needed. As
an example, a daemon using Bluetooth would rely on the built-in
mechanisms to attach devices such that they recognize each other when
placed at proximity.

\paragraph {Update Logging:}
pfsd needs to maintain a log of every update it sees in order to
retransmit them to other devices. Depending on the context, updates
should be augmented by pfsd with a propagation vector to keep track of
which devices have already receive any given update. Updates whose
version is dominated by more recent updates have to be
elided. Moreover, any update that has been propagated to all devices
can and should be removed from the log to avoid wasting local storage
capacity. This prevents simple retransmission of the whole update
log when a new device joins a \emph{file system}. Nevertheless such
initial synchronization can be achieved by transmitting a back-end
storage archive from any other \emph{replica}.

\paragraph {File transmission:}
Along with the updates, back-end storage files containing the
immutable content of file verisons handled by pFS have to be
transmitted between the \emph{replicas}. Files associated with
different versions of a same file are very likely to have
similarities. rsync\cite{tridgell:rsync} principles as well as concepts
introduced in LBFS\cite{muthitacharoen:lbfs} should definitely be
considered for file transmission in order to avoid wasting bandwidth.

\paragraph {File System management:}
\emph{file system} management is a real challenge in pFS. libpfs
assumes that the list of \emph{replicas} participating in a \emph{file
  system} is stored in the \emph{file system} tree itself. This allows
propagation of up-to-date \emph{replica} information using the pFS
mechanisms. Any communication between two devices should check that
the destination device already knows the existence of the \emph{file
  system} to be updated. If not, a special command should be issued to
instruct the destination pfsd to initiate a \emph{replica}
locally. 

A \emph{replica} removal can be done simply by updating the {\tt
  .pfs\_replica} file. The device removed would then stop receiving
updates from that \emph{file system}. Even if not needed for
correctness, it would be valuable for the ID of the removed
\emph{replica} was erased from any version vector existing in the
system. This functionality should be provided by libpfs and the
trigger for this action propagated as an update by pfsd.

\paragraph {Choice of communication channel:}
If different communication channels are available (IP (LAN / WAN),
Bluetooth, USB), it is essential to define a policy to decide which
one to use given the updates to be propagated. It would be inefficient
to propagate GB of music over WAN from a desktop computer at home to a
desktop computer at work if this data can be efficiently pushed to a
mobile device such as a digital player that will be able to quickly
retransmit it when connected to the desktop computer at work. 

There is a need for the formalization and the automatic acquisition of
statistics concerning file access patterns, connectivity patterns, and
propagations dynamics in the context of pFS. such statistics could be
exploited to make efficient choice of the communication chanels to
use given the context in which the updates have to be propagated.

\paragraph {Limited storage capacity}
Such statistics could be exploited to elaborate flexible yet efficient
caching policies in the context of limited storage capacity on mobile
devices. By flexible, we mean that users and applications should be
able to specify the caching policy at a fine grain level, based on
their more complete semantic knowledge of the data pFS is handling. By
efficient, we mean that the caching policy, influenced by user and
applications, should still impose that updates are kept long enough on
mobile devices to ensure their propagation.


% Local Variables:
% tex-main-file: "main.ltx"
% tex-command: "make;:"
% tex-dvi-view-command: "make preview;:"
% End:
